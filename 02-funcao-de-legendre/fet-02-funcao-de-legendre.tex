\documentclass[
size=17pt,
paper=smartboard,
mode=present,
display=slidesnotes,
style=paintings,
nopagebreaks,
blackslide,
fleqn]{powerdot}
 %wj capsules prettybox
 %mode = handout or present

\usepackage{amsmath,graphicx,color}
\usepackage[brazilian]{babel}
\usepackage[utf8]{inputenc}

\pdsetup{
   lf = {TC01001-FET},
   rf = {Equação de Legendre},palette = {GoldenGate}, randomdots={false}
}
% 

%opening
\title{\Large TC01001 -- Funções Especiais para Telecomunicações\\ \vspace{1cm}Equação de Legendre}
\author{Ronaldo de Freitas Zampolo\\FCT-ITEC-UFPA}
\date{ }

\begin{document}
   \maketitle[randomdots={false}]
   \begin{slide}{Agenda}
      \tableofcontents[content=sections]
   \end{slide}

   \section[ slide = true]{Solução da equação de Legendre}
      \begin{slide}[toc=]{Solução em séries de potências em um ponto ordinário}
         \begin{itemize}
            \item Equações homogêneas
		    \begin{equation}
			    y''+P(x)y' + Q(x)y = 0
		    \end{equation}
			 Se $x_0$ é um ponto ordinário da equação, então a solução geral em um intervalo contendo $x_0$ é
		    \begin{align}
			    y(x) &= \sum_{n=0}^\infty a_n(x-x_0)^n\\
				 &= a_0y_1(x) + a_1 y_2(x)
		    \end{align}
		    onde $a_0$ e $a_1$ são constantes arbitrárias e $y_1(x)$ e $y_2(x)$ são funções analíticas em $x_0$, linearmente independentes.
         \end{itemize}
      \end{slide}
      
      \begin{slide}[toc=]{Equação de Legendre}
         \begin{itemize}
            \item Equação de Legendre
		    \begin{equation}
			    (1-x^2)y''- 2xy' + \alpha(\alpha+1)y = 0
		    \end{equation}
	    \item[1º] passo: verificar se $x=0$ é um ponto ordinário da equação diferencial
		    \begin{enumerate}
			    \item Colocar a equação na forma canônica:
			    \item Avaliar os limites de $P(x)$ e $Q(x)$ quando $x$ tende a zero
		            \item Se o limite não tender ao infinito, a solução geral pode ser representada por uma série de potência
		    \end{enumerate}
	    \item[2º] passo: representar $y(n)$ como uma série de potência
	    \item[3º] passo: substituir na equação de Legendre
	    \item[4º] passo: organizar e agrupar os termos em comum
	    \item[5ª] passo: obter as fórmulas de recorrência
         \end{itemize}         
      \end{slide}
      
      \begin{slide}[toc=]{Fórmula de recorrência}
        \begin{itemize}
		\item Coeficientes da série de potências
			\begin{equation}
				a_{n+2} = \frac{(n-\alpha)(n+\alpha+1)}{(n+2)(n+1)} a_n
			\end{equation}
		\item O que acontece quando $n=\alpha$ ?
          \end{itemize}
       \end{slide}
  
   \section[ slide = false ]{Exercícios e aplicações}
      \begin{slide}[toc=]{Exercícios}
         \begin{enumerate}
            \item Trace gráficos dos polinômios de Legendre para $\alpha=0,1,2,\dots, 8$
	    \item Descubra três aplicações associadas à equação/aos polinômios de Legendre
            \end{enumerate}
      \end{slide}
      
\end{document}
