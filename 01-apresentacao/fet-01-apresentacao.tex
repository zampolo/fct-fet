\documentclass[
size=17pt,
paper=smartboard,
mode=present,
display=slidesnotes,
style=sailor,
nopagebreaks,
blackslide,
fleqn]{powerdot}
 %wj capsules prettybox
 %mode = handout or present

\usepackage{amsmath,graphicx,color}
\usepackage[brazilian]{babel}
\usepackage[utf8]{inputenc}

\pdsetup{
   lf = {EC01045 PDS},
   rf = {Apresentação do curso},palette = {Sea}, randomdots={false}
}
% 

%opening
\title{\Large EC01045 -- Processamento Digital de Sinais\\ \vspace{1cm}Apresentação do curso}
\author{Ronaldo de Freitas Zampolo\\FCT-ITEC-UFPA}
\date{ }

\begin{document}
   \maketitle[randomdots={false}]
   \begin{slide}{Agenda}
      \tableofcontents[content=sections]
   \end{slide}

   \section[ slide = false ]{Professor }
      \begin{slide}[toc=]{Professor}
         \begin{itemize}
            \item Professor: Ronaldo de Freitas Zampolo
            \item Afiliação:\\
                  Laboratório de Processamento de Sinais - LaPS\\
                  Faculdade de Eng. da Computação e Telecomunicações - FCT\\
                  Instituto de Tecnologia - ITEC\\
                  Universidade Federal do Pará - UFPA
            \item Atendimento:\\
                  Terça-feira: 13h30 - 14h30\\
                  Sala~32, anexo do LEEC\\
                  \texttt{zampolo@ufpa.br}\\ 
                  \texttt{zampolo@ieee.org}\\
                  \texttt{www.laps.ufpa.br/zampolo}
         \end{itemize}
      \end{slide}
      
   \section[ slide = false ]{Características do Curso}
      \begin{slide}[toc=]{Características do Curso}
         \begin{itemize}
            \item Carga horária: 60 h
            \item Aulas: terças e quintas (das 16h40 às 18h20)
            \item Tópicos:
            \begin{itemize}
               \item Sinais e sistemas discretos (Cap. 2)
               \item Transformada Z (Cap. 3)
               \item Amostragem de sinais contínuos (Cap. 4)
               \item Análise de sistemas lineares invariantes no domínio transformado (Cap. 5)
               \item Estruturas para implementação de sistemas discretos (Cap. 6)
               \item Técnicas para projetos de filtros (Cap. 7)
               \item Transformada de Fourier discreta (Cap. 8)
            \end{itemize}
         \end{itemize}         
      \end{slide}
      
%       \begin{slide}[toc=]{Plataforma Moodle}
%          \begin{itemize}
%             \item URL: \texttt{http://www.aedmoodle.ufpa.br/ }
%             \begin{itemize}
%                \item \emph{Downloads}
%                \item Avaliações
%                \item Avisos
%                \item \emph{Links}
%             \end{itemize}
%             \item Observação: o curso é \emph{presencial}!
%          \end{itemize}
%       \end{slide}
%   
   \section[slide=false]{Objetivos do curso}
      \begin{slide}[toc=]{Objetivos do curso}
         \begin{itemize}
            \item Obter familiaridade:
            \begin{itemize}
               \item Matemática
               \item Conceitos
            \end{itemize}
            \item Utilizar ferramentas básicas
            \begin{itemize}
               \item Projeto
               \item Simulação
               \item Implementação
            \end{itemize}
            
         \end{itemize}
      \end{slide}

   \section[slide=false]{Aplicações}
      \begin{slide}[toc=]{Algumas situações de aplicação}
         \begin{itemize}
            \item Telecomunicações: equalizadores de canal, repetidores digitais, moduladores/demoduladores...
            \item Processamento de imagem/vídeo: restauração, codificação, avaliação de qualidade, segmentação, realce...
            \item Processamento de fala: identificação de locutor, reconhecimento, conversores texto-fala e fala-texto...
            \item Outros tipos de sinais: sísmico, temperatura, pressão...
         \end{itemize}
      \end{slide}


  
   \section[ slide = false ]{Bibliografia}
      \begin{slide}[toc=]{Bibliografia}
         \begin{itemize}
            \item Bibliografia básica
            \begin{itemize}
               \item \textbf{A. V. Oppenheim, R. W. Schafer. \emph{Processamento 
em tempo discreto de sinais}, 3ª Ed., Pearson, 2013.}
            \end{itemize}
            \item Bibliografia complementar
            \begin{itemize}
               \item P. S. R. Diniz, E. A. B. Silva, S. L. Netto. \emph{Processamento 
               Digital de Sinais - Projeto e análise de sistemas}, 2ª edição, Bookman Company, 2014.
               \item A. V. Oppenheim, A. S. Willsky. \emph{Sinais e Sistemas}, 2ª edição, Pearson, 2010.
            \end{itemize}
         \end{itemize}
      \end{slide}

   \section[ slide = false ]{Avaliação}
      \begin{slide}[toc=]{Instrumentos}
         \begin{itemize}
            \item Avaliação continuada
            %\item %Exercícios em sala (Es): exercícios rápidos (5 a 10 min em quase todas as aulas) 
            %\item %Listas de exercício (Le): entregues na última aula da semana (prazo de entrega de uma semana)
            \item Trabalhos (Tr): em equipe ou individual com apresentação à turma
            \item Testes escritos (Te): em número de 3; 2ª chamada mediante requerimento
            \item Cálculo da nota ($N$):
            \begin{equation*}
               N=\frac{( \text{Tr} \times 4 + \text{Te} \times 6 )} {10}
            \end{equation*}
          \end{itemize}
      \end{slide}
      
      \begin{slide}[toc=]{Mapeamento nota-conceito}
         \begin{itemize}
            \item Cálculo da nota ($N$): 
            \begin{equation*}
               N=\frac{( \text{Tr} \times 4 + \text{Te} \times 6 )} {10}
            \end{equation*}
            \item Tabela de mapeamento:
            \begin{table}
               \centering
               \begin{tabular}{c|c}
                  \hline\hline
                  \textbf{Faixa} & \textbf{Conceito}\\
                  \hline
                  0 -- 4,9 & INS\\
                  5,0 -- 6,9 & REG\\
                  7,0 -- 8,9 & BOM\\
                  9,0 -- 10,0 & EXC\\
                  \hline\hline
               \end{tabular}
            \end{table}    
         \end{itemize}
      \end{slide}
      
      \begin{slide}[toc=]{Datas dos testes escritos}
         \begin{itemize}
            \item Datas prováveis dos testes escritos:
            \begin{table}
               \centering
               \begin{tabular}{|l l|}
                  \hline
                  Teste 01: & 12/setembro\\
                  Teste 02: & 05/novembro\\
                  Teste 03: & 12/dezembro\\
                  %Teste 04: & 13/dezembro\\
                  %Substitutiva: & a definir\\
                  \hline
               \end{tabular}
            \end{table}
         \end{itemize}
      \end{slide}
%   \section[ slide = false ]{Código de conduta}
%      \begin{slide}[toc=]{Código de conduta}
%      \footnotesize
%      Todos os envolvidos devem contribuir para que o ambiente em sala de aula seja o mais propício possível para que ocorram o aprendizado e o desenvolvimento pessoal em bases de respeito, justiça e honestidade.
%         \begin{itemize}
%            \item É de responsabilidade de cada um configurar seu \textbf{telefone celular} para o modo silencioso (ou equivalente), atendendo-o apenas em casos de real necessidade, e de maneira a não perturbar a aula;
%            \item O uso de \textbf{laptops}, \textbf{tablets} e o \textbf{acesso à internet} devem se restringir, durante as aulas, às atividades acadêmicas quando e se solicitado pelo docente responsável;
%            \item A \textbf{atenção} deve estar concentrada na atividade proposta ou exposição realizada pelo instrutor. Comportamentos incompatíveis, tais como realizar tarefas de outras disciplinas, ou trabalhos atrasados, conversas paralelas, e dormir devem ser evitados;
%            \item Conservar a \textbf{sala de aula limpa};
%            \item É imprescindível manter a \textbf{integridade} do processo de verificação de aprendizagem. Atos como cópia de trabalho, fraude, fazer-se passar por outra pessoa em momentos de avaliação, bem como o uso de ajudas não autorizadas de qualquer tipo, violam princípios elementares de honestidade e decoro acadêmico e, portanto, não serão toleradas.
%         \end{itemize}
%      \end{slide}
\end{document}
