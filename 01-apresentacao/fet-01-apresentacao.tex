\documentclass[
size=17pt,
paper=smartboard,
mode=present,
display=slidesnotes,
style=paintings,
nopagebreaks,
blackslide,
fleqn]{powerdot}

% styles: sailor, paintings
% wj capsules prettybox
% mode = handout or present


\newcommand{\palette}{PearlEarring}
% palettes:
%    - sailor: Sea, River, Wine, Chocolate, Cocktail 
%    - paintings: Syndics, Skater, GoldenGate, Moitessier, PearlEarring, Lamentation, HolyWood, Europa, MayThird, Charon 


\usepackage{amsmath,graphicx,color,amsfonts}
\usepackage[brazilian]{babel}
\usepackage[utf8]{inputenc}
\usepackage{bbding}

\newcommand{\cursopequeno}{TC01001 FET}
\newcommand{\cursogrande}{\Large TC01001 -- Funções especiais em telecomunicações}
\newcommand{\alert}[1]{\textcolor{red}{#1}}
\newtheorem{theorem}{Teorema}
\author{Ronaldo de Freitas Zampolo\\FCT-ITEC-UFPA}
\date{2022-2}


\pdsetup{
   lf = {\cursopequeno},
   rf = {Apresentação do curso}, palette = {\palette}, randomdots={false},
   cf = {\theslide}
}

\title{\cursogrande\\ \vspace{1cm}Apresentação do curso}

\begin{document}
   \maketitle[randomdots={false}]
   \begin{slide}{Agenda}
      \tableofcontents[content=sections]
   \end{slide}

   \section[ slide = true ]{Professor }
      \begin{slide}[toc=]{Professor}
         \begin{itemize}
            \item Professor: Ronaldo de Freitas Zampolo
            \item Afiliação:\\
                  Laboratório de Processamento de Sinais - LaPS\\
                  Faculdade de Eng. da Computação e Telecomunicações - FCT\\
                  Instituto de Tecnologia - ITEC\\
                  Universidade Federal do Pará - UFPA
            \item Contato:\\
                  %Terça-feira: 14h00 - 14h50\\
                  %Sala~32, anexo do LEEC\\
                  \texttt{zampolo@ufpa.br}\\ 
                  %\texttt{zampolo@ieee.org}\\
                  %\texttt{www.laps.ufpa.br/zampolo}
         \end{itemize}
      \end{slide}
      
   \section[ slide = true ]{Características do Curso}
      \begin{slide}[toc=]{Características do Curso}
         \begin{itemize}
            \item Carga horária: 60 h
            \item Aulas: segundas e quartas às 16h40
            \item Tópicos:
            \begin{itemize}
               \item Funções especiais: Legendre
               %\item Equações Fuchsianas
               \item Função hipergeométrica
               \item Polinômios ortogonais
	       \item Sistemas de Sturm-Liouville
               \item Expansão em autofunções
               \item Séries de Fourier
	       \item Séries generalizadas
            \end{itemize}
         \end{itemize}         
      \end{slide}
      
   \section[ slide = true ]{Bibliografia}
      \begin{slide}[toc=]{Bibliografia}
         \begin{itemize}
            \item Bibliografia básica
            \begin{itemize}
		    \item \textbf{K. T. Tang. \emph{Mathematical methods for Engineers and scientists 3: Fourier analysis, partial differential equations and variational methods}, Springer, 2007.}
               \item {G. B. Arfken, H. J. Weber. \emph{Física Matemática -- métodos matemáticos para engenharia e física}, 6ª Ed., Campus, 2007.}
               \item {E. Butkov. \emph{Física Matemática}, Guanabara Koogan, 1988.}
               \item {R. Bronson. \emph{Moderna introdução às equações diferenciais}, McGraw-Hill, 1976.}
            \end{itemize}
         \end{itemize}
      \end{slide}

\section[slide=true]{Habilidades e competências}
      \begin{slide}[toc=]{Habilidades competências}
         \begin{itemize}
           % \item Objetivos:
	%	    \begin{itemize}
	%		    \item Conhecer as funções de Legendre e suas aplicações
	%		    \item Conhecer a função hipergeométrica e suas aplicações
	%		    \item Entender o conceito de autofunções
	%		    \item Aplicar o conceito de autofunções na resolução de equações diferenciais e decomposição de funções
	%		    \item Aplicar e interpretar a decomposição de funções em série de Fourier
	%		    \item Generalizar a representação de uma função por expansão em séries
	%	    \end{itemize}
	     \item Habilidades e competências:
		     \begin{itemize}
			     \item Conhecimento sobre as funções de Legendre e suas áreas de aplicação
			     \item Conhecimento sobre a função hipergeométrica e suas áreas de aplicação
			     \item Entendimento sobre o conceito de autofunções
			     \item Resolução de equações diferenciais usando autofunções
			     \item Aplicação e interpretação da decomposição de funções usando série de Fourier
			     \item Representação de funções por expansão em séries
	            \end{itemize}
         \end{itemize}
      \end{slide}

    \section[ slide = true]{Metodologia, ferramentas e avaliação}
      \begin{slide}[toc=]{Metodologia, ferramentas e avaliação}
         \begin{itemize}
	    \item Metodologia utilizada: aula invertida  
		    \begin{itemize}
			    \item Atividades em sala de aula: resolução de exercícios, aprofundamento do conteúdo
			    %\item Chat no SIGAA (atividade síncrona): plantão de dúvidas
			    \item Maior parte do estudo será feita de maneira guiada, mas assíncrona: leituras, vídeos, atividades diversas
		    \end{itemize}
	    \item Ferramentas usadas:
		    \begin{itemize}
			    \item Atividades remotas: Google Meet %RNP e SIGAA (Chat)
			    \item SIGAA (repositório e sistema de entrega)
			    \item Simulações, programação: Google Colaboratory (Python)
		    \end{itemize}
            \item Avaliação continuada
            \begin{itemize}
               \item Trabalhos 
               \item Tarefas 
	       \item Listas de exercício
	    \end{itemize}
	 \end{itemize}
	 \end{slide}	  
   \section[ slide = true ]{Avaliação}
      \begin{slide}[toc=]{Instrumentos}
         \begin{itemize}
            %\item Avaliação continuada
            %\item %Exercícios em sala (Es): exercícios rápidos (5 a 10 min em quase todas as aulas) 
            \item Listas de exercício/atividades (LA)
            %\item Trabalhos (Tr): em equipe ou individual com apresentação à turma
            \item Prova (PR)
            \item Cálculo da nota ($N$):
            \begin{equation*}
               N=\frac{( \text{LA} \times 4+ \text{PR} \times 6 )} {10}
            \end{equation*}
            \item Tabela de mapeamento:
            \begin{table}
               \centering
               \begin{tabular}{c|c}
                  \hline%\hline
                  \textbf{Faixa} & \textbf{Conceito}\\
                  \hline
                  0 -- 4,9 & INS\\
                  5,0 -- 6,9 & REG\\
                  7,0 -- 8,9 & BOM\\
                  9,0 -- 10,0 & EXC\\
                  \hline%\hline
               \end{tabular}
            \end{table}    
            \item Data provável das provas: 21 de novembro e 19 de dezembro
           \end{itemize}
      \end{slide}
%      
%      \begin{slide}[toc=]{Mapeamento nota-conceito}
%         \begin{itemize}
%            \item Cálculo da nota ($N$): 
%            \begin{equation*}
%               N=\frac{( \text{LA} \times 4 + \text{Te} \times 6 )} {10}
%            \end{equation*}
%            \item Tabela de mapeamento:
%            \begin{table}
%               \centering
%               \begin{tabular}{c|c}
%                  \hline%\hline
%                  \textbf{Faixa} & \textbf{Conceito}\\
%                  \hline
%                  0 -- 4,9 & INS\\
%                  5,0 -- 6,9 & REG\\
%                  7,0 -- 8,9 & BOM\\
%                  9,0 -- 10,0 & EXC\\
%                  \hline%\hline
%               \end{tabular}
%            \end{table}    
%            \item Data provável da prova: 05/julho
%         \end{itemize}
%      \end{slide}
%      
%      \begin{slide}[toc=]{Datas dos testes escritos}
%         \begin{itemize}
%            \item Data provável da prova:
%            \begin{table}
%               \centering
%               \begin{tabular}{|l l|}
%                  \hline
%                  Teste 01: & 05/julho\\
%                  Teste 02: & 11/dezembro\\
                  %Teste 04: & 13/dezembro\\
                  %Substitutiva: & a definir\\
%                  \hline
%               \end{tabular}
%            \end{table}
%         \end{itemize}
%      \end{slide}
%%   \section[ slide = false ]{Código de conduta}
%      \begin{slide}[toc=]{Código de conduta}
%      \footnotesize
%      Todos os envolvidos devem contribuir para que o ambiente em sala de aula seja o mais propício possível para que ocorram o aprendizado e o desenvolvimento pessoal em bases de respeito, justiça e honestidade.
%         \begin{itemize}
%            \item É de responsabilidade de cada um configurar seu \textbf{telefone celular} para o modo silencioso (ou equivalente), atendendo-o apenas em casos de real necessidade, e de maneira a não perturbar a aula;
%            \item O uso de \textbf{laptops}, \textbf{tablets} e o \textbf{acesso à internet} devem se restringir, durante as aulas, às atividades acadêmicas quando e se solicitado pelo docente responsável;
%            \item A \textbf{atenção} deve estar concentrada na atividade proposta ou exposição realizada pelo instrutor. Comportamentos incompatíveis, tais como realizar tarefas de outras disciplinas, ou trabalhos atrasados, conversas paralelas, e dormir devem ser evitados;
%            \item Conservar a \textbf{sala de aula limpa};
%            \item É imprescindível manter a \textbf{integridade} do processo de verificação de aprendizagem. Atos como cópia de trabalho, fraude, fazer-se passar por outra pessoa em momentos de avaliação, bem como o uso de ajudas não autorizadas de qualquer tipo, violam princípios elementares de honestidade e decoro acadêmico e, portanto, não serão toleradas.
%         \end{itemize}
%      \end{slide}
\end{document}
