\documentclass[
size=17pt,
paper=smartboard,
mode=present,
display=slidesnotes,
style=paintings,
nopagebreaks,
blackslide,
fleqn]{powerdot}
 %wj capsules prettybox
 %mode = handout or present

\usepackage{amsmath,graphicx,color}
\usepackage[brazilian]{babel}
\usepackage[utf8]{inputenc}

\pdsetup{
   lf = {TC01001-FET},
   rf = {Equações Fuchsianas},palette = {GoldenGate}, randomdots={false}
}
% 

%opening
\title{\Large TC01001 -- Funções Especiais para Telecomunicações\\ \vspace{1cm}Equações Fuchsianas}
\author{Ronaldo de Freitas Zampolo\\FCT-ITEC-UFPA}
\date{ }

\begin{document}
   \maketitle[randomdots={false}]
   \begin{slide}{Agenda}
      \tableofcontents[content=sections]
   \end{slide}

   \section[ slide = true]{Equações diferenciais hipergeométricas}
      \begin{slide}[toc=]{Série hipergeométrica}
         \begin{itemize}
		 \item Para qualquer número complexo $a$ e inteiro $n$, define-se $(a,n)$ como sendo o produto 
		    \begin{equation}
			    (a,n) = a (a+1) \cdots (a+n-1) = \frac{\Gamma(a+n)}{\Gamma(a)}
		    \end{equation}
		\item Função Gama:
			\begin{equation}
				\Gamma(z) = \int_0^\infty x^{z-1}e^{-x}dx, \quad \Re(z)>0
			\end{equation}
				
		 \item Série hipergeométrica
		    \begin{equation}
			    F(a,b,c;x) = \sum_{n=0}^\infty \frac{(a,n)(b,n)}{(c,n)(1,n)}x^n = \sum_{n=0}^\infty A_nx^n
		    \end{equation}
			 onde $a$, $b$ e $c$ ($c\neq 0, -1, -2, \cdots$) são números complexos.
         \end{itemize}
      \end{slide}
      
      \begin{slide}[toc=]{Equações diferenciais hipergeométricas}
         \begin{itemize}
		 \item Equação diferencial cuja solução é $F(a,b,c;x)$
		    \begin{equation}
			    x (1-x)y''+[c-(a+b+1)x]y' - aby = 0
		    \end{equation}
         \end{itemize}         
      \end{slide}
      
      \begin{slide}[toc=]{Fórmula de recorrência}
        \begin{itemize}
		\item Coeficientes da série de potências
			\begin{equation}
				a_{n+2} = \frac{(n-\alpha)(n+\alpha+1)}{(n+2)(n+1)} a_n
			\end{equation}
		\item O que acontece quando $n=\alpha$ ?
          \end{itemize}
       \end{slide}
  
   \section[ slide = false ]{Exercícios e aplicações}
      \begin{slide}[toc=]{Exercícios}
         \begin{enumerate}
		 \item  Mostre que $\Gamma(z+1) = z\Gamma(z)$.
		 \item  Se $z$ for um número inteiro positivo, demonstre que $\Gamma(z) = (z-1)!$
		 \item  Usando a definição da função Gama, mostre que $0!=1$
            \end{enumerate}
      \end{slide}
      
\end{document}
