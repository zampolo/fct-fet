\documentclass[
size=17pt,
paper=smartboard,
mode=present,
display=slidesnotes,
style=paintings,
nopagebreaks,
blackslide,
fleqn]{powerdot}

% styles: sailor, paintings
% wj capsules prettybox
% mode = handout or present


\newcommand{\palette}{PearlEarring}
% palettes:
%    - sailor: Sea, River, Wine, Chocolate, Cocktail 
%    - paintings: Syndics, Skater, GoldenGate, Moitessier, PearlEarring, Lamentation, HolyWood, Europa, MayThird, Charon 


\usepackage{amsmath,graphicx,color,amsfonts}
\usepackage[brazilian]{babel}
\usepackage[utf8]{inputenc}
\usepackage{bbding}

\newcommand{\cursopequeno}{TC01001 FET}
\newcommand{\cursogrande}{\Large TC01001 -- Funções especiais em telecomunicações}
\newcommand{\alert}[1]{\textcolor{red}{#1}}
\newtheorem{theorem}{Teorema}
\author{Ronaldo de Freitas Zampolo\\FCT-ITEC-UFPA}
\date{2022-2}


\pdsetup{
   lf = {\cursopequeno},
   rf = {Teoria de Sturm-Liouville}, palette = {\palette}, randomdots={false},
   cf = {\theslide}
}

\title{\cursogrande\\ \vspace{1cm}Teoria de Sturm-Liouville}

\begin{document}
   \maketitle[randomdots={false}]
   \begin{slide}{Agenda}
      \tableofcontents[content=sections]
   \end{slide}

   \section[ slide = true ]{Equações de Sturm-Liouville}
   \begin{slide}[toc=]{Equações de Sturm-Liouville}
		\begin{itemize}
			\item Tome-se como ponto de partida a equação diferencial linear de segunda ordem:
				\begin{equation*}
					A(x)\frac{d^2y(x)}{dx^2} + B(x)\frac{dy(x)}{dx}+ C(x)y(x)+\lambda D(x)y(x) = 0 
				\end{equation*}
					onde $\lambda$ é um parâmetro a ser determinada pelas condições de contorno do problema.
			\item Uma forma alternativa da equação acima seria:
				\begin{equation*}
					\frac{d^2y(x)}{dx^2} + b(x)\frac{dy(x)}{dx}+ c(x)y(x)+\lambda d(x)y(x) = 0 
				\end{equation*}
				onde $b(x) = \frac{B(x)}{A(x)}$, $c(x) = \frac{C(x)}{A(x)}$, e $d(x) = \frac{D(x)}{A(x)}$, com $A(x)\neq 0$.
		\end{itemize}
      \end{slide}
      
      \begin{slide}[toc=]{Equações de Sturm-Liouville}
		\begin{itemize}
			\item Seja definido o fator integrante $p(x)$
				\begin{equation*}
					p(x) = e^{\int^x b(x^\prime)\,dx^\prime} 
				\end{equation*}
			\item Multiplicando a equação normalizada por $p(x)$, tem-se 
				\begin{equation*}
					p(x)\frac{d^2y(x)}{dx^2} + p(x)b(x)\frac{dy(x)}{dx}+ p(x)c(x)y(x)+\lambda p(x)d(x)y(x) = 0 
				\end{equation*}
			\item Contudo:
						\begin{align*}
							\frac{dp(x)}{dx} &= \frac{d}{dx}e^{\int^x b(x^\prime)\,dx^\prime}\\
									 &= e^{\int^x b(x^\prime)\,dx^\prime}\frac{d}{dx}\int^x b(x^\prime)\,dx^\prime\\
									 &= p(x)b(x)
						\end{align*}
		\end{itemize}
      \end{slide}
   
      \begin{slide}[toc=]{Equações de Sturm-Liouville}
		\begin{itemize}
			\item Seja definido o fator integrante $p(x)$
				\begin{equation*}
					p(x) = e^{\int^x b(x^\prime)\,dx^\prime} 
				\end{equation*}
			\item Multiplicando a equação normalizada por $p(x)$, tem-se 
				\begin{equation*}
					p(x)\frac{d^2y(x)}{dx^2} + p(x)b(x)\frac{dy(x)}{dx}+ p(x)c(x)y(x)+\lambda p(x)d(x)y(x) = 0 
				\end{equation*}
			\item Contudo:
						\begin{align*}
							\frac{dp(x)}{dx} &= p(x)b(x)\\
							\frac{d}{dx}\left [p(x)\frac{dy(x)}{dx} \right ] &= \frac{dp(x)}{dx}\frac{dy(x)}{dx}+p(x)\frac{d^2y(x)}{dx^2}\\
									 &= p(x)\frac{d^2y(x)}{dx^2}+ p(x)b(x)\frac{dy(x)}{dx}
						\end{align*}
		\end{itemize}
      \end{slide}

      \begin{slide}[toc=]{Equações de Sturm-Liouville}
		\begin{itemize}
			\item Assim, a equação  
				\begin{equation*}
					p(x)\frac{d^2y(x)}{dx^2} + p(x)b(x)\frac{dy(x)}{dx}+ p(x)c(x)y(x)+\lambda p(x)d(x)y(x) = 0 
				\end{equation*}
				pode ser rescrita como (equação de Sturm-Liouville):
				\begin{equation*}
					\frac{d}{dx}\left [p(x)\frac{dy(x)}{dx} \right ]+ q(x)y(x)+\lambda w(x)y(x) = 0 
				\end{equation*}
				onde:
				\begin{align*}
					q(x) &= p(x)c(x)\\
					w(x) &= p(x)d(x)
				\end{align*}
			\item Condições:
				\begin{itemize}
					\item $p(x)$, $q(x)$, $w(x)$ são funções reais e contínuas;
					\item $p(x)$, e $w(x)$ são positivas em um dado intervalo.
				\end{itemize}
		\end{itemize}
      \end{slide}

      \begin{slide}[toc=]{Equações de Sturm-Liouville}
		\begin{itemize}
			\item As equações de Sturm-Liouville podem ser postas da forma do problema de autovalores:
				\begin{equation*}
					Ly(x) = \lambda y(x)
				\end{equation*}
				mediante a definição do operador de Sturm-Liouville:
				\begin{equation*}
					L = -\frac{1}{w(x)}\left [ \frac{d}{dx}\left ( p(x)\frac{d}{dx} \right ) + q(x) \right ]
				\end{equation*}
			\item Ou seja 
				\begin{align*}
					\frac{d}{dx}\left [p(x)\frac{dy(x)}{dx} \right ]+ q(x)y(x)+\lambda w(x)y(x) &= 0 \\
					\frac{d}{dx}\left [p(x)\frac{dy(x)}{dx} \right ]+ q(x)y(x)&=-\lambda w(x)y(x) \\
					-\frac{1}{w(x)}\left \{\frac{d}{dx}\left [p(x)\frac{dy(x)}{dx} \right ]+ q(x)y(x)\right \} &=\lambda y(x)
				\end{align*}
		\end{itemize}
      \end{slide}
   
   \section[ slide = true ]{Condições de contorno de problemas de Sturm-Liouville}
      \begin{slide}[toc=]{Condições de contorno de problemas de Sturm-Liouville}
		\begin{itemize}
			\item Os operadores de Sturm-Liouville são operadores Hermitianos:
				\begin{align*}
					<Lf(x),g(x)> &= \int_a^b\left \{ -\frac{1}{w(x)}\left [ \frac{d}{dx}\left ( p(x)\frac{d}{dx} \right ) + q(x) \right ]f(x)\right \}^\ast g(x) w(x)\,dx\\
					&= -\int_a^b \frac{d}{dx}\left ( p(x)\frac{df^\ast(x)}{dx} \right ) g(x)\,dx -\int_a^b q(x) f^\ast(x)g(x)\,dx
				\end{align*}
				Lembrando que $p(x)$, $q(x)$, e $w(x)$ são funções reais.
			\item Considerando que
				\begin{align*}
					\int_a^b \frac{d}{dx}\left ( p(x)\frac{df^\ast(x)}{dx} \right ) g(x)\,dx &= p(x)\frac{df^\ast(x)}{dx}g(x)\Big \vert_a^b - \int_a^bp(x)\frac{df^\ast (x)}{dx}\frac{dg(x)}{dx} dx\\
					\int_a^bp(x)\frac{df^\ast (x)}{dx}\frac{dg(x)}{dx} dx &= \int_a^b\frac{df^\ast (x)}{dx}p(x)\frac{dg(x)}{dx} dx\\
					&=f^\ast (x)p(x)\frac{dg(x)}{dx}\Big \vert_a^b - \int_a^bf^\ast (x)\frac{d}{dx}\left (p(x)\frac{dg(x)}{dx}\right ) dx
				\end{align*}
		\end{itemize}
      \end{slide}

      \begin{slide}[toc=]{Condições de contorno de problemas de Sturm-Liouville}
		\begin{itemize}
			\item Juntando tudo, tem-se
				\begin{align*}
					<Lf(x),g(x)> =& - p(x)\frac{df^\ast(x)}{dx}g(x)\Big \vert_a^b+f^\ast (x)p(x)\frac{dg(x)}{dx}\Big \vert_a^b +\\
					              &- \int_a^bf^\ast (x)\frac{d}{dx}\left (p(x)\frac{dg(x)}{dx}\right ) dx-\int_a^b q(x) f^\ast(x)g(x)\,dx\\
						      =&  p(x)\left [ f^\ast (x)\frac{dg(x)}{dx}-\frac{df^\ast(x)}{dx}g(x)\right ]\Bigg \vert_a^b + \\
						       &+\int_a^bf^\ast (x)\left \{-\frac{1}{w(x)} \left [\frac{d}{dx}\left (p(x)\frac{d}{dx}\right ) + q(x)\right ]g(x)\right \} w(x)\,dx\\
						      =&p(x)\left [ f^\ast (x)\frac{dg(x)}{dx}-\frac{df^\ast(x)}{dx}g(x)\right ]\Bigg \vert_a^b + <f(x),Lg(x)>
				\end{align*}
%			\item Se o espaço de funções satisfaz
%				\begin{equation*}
%					p(x)\left [ f^\ast (x)\frac{dg(x)}{dx}-\frac{df^\ast(x)}{dx}g(x)\right ]\Bigg \vert_a^b = 0
%				\end{equation*}
%				Então $<Lf(x),g(x)> = <f(x),Lg(x)>$, ou seja, o operador de Sturm-Liouville é Hermitiano.
%
		\end{itemize}
      \end{slide}      
      
      \begin{slide}[toc=]{Condições de contorno de problemas de Sturm-Liouville}
		\begin{itemize}
			\item Repetindo o resultado principal do slide anterior
				\begin{align*}
					<Lf(x),g(x)> =p(x)\left [ f^\ast (x)\frac{dg(x)}{dx}-\frac{df^\ast(x)}{dx}g(x)\right ]\Bigg \vert_a^b + <f(x),Lg(x)>
				\end{align*}
			\item Se o espaço de funções satisfaz
				\begin{equation*}
					p(x)\left [ f^\ast (x)\frac{dg(x)}{dx}-\frac{df^\ast(x)}{dx}g(x)\right ]\Bigg \vert_a^b = 0
				\end{equation*}
				Então $<Lf(x),g(x)> = <f(x),Lg(x)>$, ou seja, o operador de Sturm-Liouville é Hermitiano.
			\item O que se chama de \emph{Problemas de Sturm-Liouville} consiste na equação de Sturm-Liouville, mais o conjunto de condições de contorno.

		\end{itemize}
      \end{slide}
   
      \begin{slide}[toc=]{Condições de contorno de problemas de Sturm-Liouville}
		\begin{itemize}
			\item Uma vez que o operador de Sturm-Liouville seja Hermitiano, as autofunções do problema de Sturm-Liouville são ortogonais entre si e formam uma base completa.
			\item Sejam $y_m(x)$ e $y_n(x)$ soluções da equação diferencial homogênea de segunda ordem 
				\begin{equation*}
					\left [ p(x)y^\prime (x)\right ]^\prime + q(x)y(x) + \lambda w(x)y(x) = 0
				\end{equation*}
				no intervalo $a\leq x\leq b$, e satisfazem as condições iniciais:
				\begin{equation*}
					p(x)\left [ f^\ast (x)\frac{dg(x)}{dx}-\frac{df^\ast(x)}{dx}g(x)\right ]\Bigg \vert_a^b = 0
				\end{equation*}
				onde $f(x) = y_n(x)$ e $g(x) = y_m(x)$.
		\end{itemize}
      \end{slide}

      \begin{slide}[toc=]{Condições de contorno de problemas de Sturm-Liouville}
		\begin{itemize}
			\item Substituindo $f(x) = y_n(x)$ e $g(x) = y_m(x)$ em 
				\begin{equation*}
					\left [ p(x)y^\prime (x)\right ]^\prime + q(x)y(x) + \lambda w(x)y(x) = 0
				\end{equation*}
				tem-se
				\begin{align*}
					p(b)\left [ y_n^\ast(b)y_m^\prime(b)-(y_n^\prime(b))^\ast(b)y_m(b)\right ] - p(a)\left [ y_n^\ast(a)y_m^\prime(a)-(y_n^\prime(a))^\ast y_m(a)\right ] =& 0\\
					p(b)\begin{vmatrix}y_n^\ast(b) & (y_n^\prime(b))^\ast\\ y_m(b) & y_m^\prime(b)\end{vmatrix}-p(a)\begin{vmatrix}y_n^\ast(a) & (y_n^\prime(a))^\ast\\ y_m(a) & y_m^\prime(a)\end{vmatrix}=& 0
				\end{align*}
			\item Considerando as funções $y_m(x)$ e $y_n(x)$ reais, 
				\begin{equation*}
					p(b)\begin{vmatrix}y_n(b) & y_n^\prime(b)\\ y_m(b) & y_m^\prime(b)\end{vmatrix}-p(a)\begin{vmatrix}y_n(a) & y_n^\prime(a)\\ y_m(a) & y_m^\prime(a)\end{vmatrix}= 0
				\end{equation*}
		\end{itemize}
      \end{slide}
   \section[ slide = true ]{Problemas de Sturm-Liouville regulares}
      \begin{slide}[toc=]{Problemas de Sturm-Liouville regulares}
	      \begin{itemize}
		      \item Considerado as condições iniciais definidas para que o operador de Sturm-Liouville $L$ seja Hermitiano:
				\begin{equation*}
					p(b)\begin{vmatrix}y_n(b) & y_n^\prime(b)\\ y_m(b) & y_m^\prime(b)\end{vmatrix}-p(a)\begin{vmatrix}y_n(a) & y_n^\prime(a)\\ y_m(a) & y_m^\prime(a)\end{vmatrix}= 0
				\end{equation*}
			\item Esta classe de problemas considera que $p(a)\neq 0$ e $p(b)\neq 0$
			\item As condições iniciais acima podem ser rescritas como
				\begin{align*}
					\alpha_1y(a) + \alpha_2 y^\prime (a) &= 0\\
					\beta_1 y(b) + \beta_2 y^\prime (b) &= 0\\
				\end{align*}
				onde $\alpha_1$ e $\alpha_2$ não são ambos iguais a zero; e $\beta_1$ e $\beta_2$ não são ambos iguais a zero
	      \end{itemize}
      \end{slide}

      \begin{slide}[toc=]{Problemas de Sturm-Liouville regulares}
	      \begin{itemize}
		      \item Supondo que $y_n(x)$ e $y_m(x)$ são soluções para o problema de Sturm-Liouville
				\begin{align*}
					\alpha_1y_n(a) + \alpha_2 y_n^\prime (a) &= 0\\
					\alpha_1y_m(a) + \alpha_2 y_m^\prime (a) &= 0
				\end{align*}
				\twocolumn{
				Como ou $\alpha_1\neq 0$ ou $\alpha_2\neq 0$, 
				\begin{equation*}
					\begin{vmatrix}y_n(a) & y_n^\prime(a)\\ y_m(a) & y_m^\prime(a)\end{vmatrix}= 0
				\end{equation*}}{
				E como ou $\beta_1\neq 0$ ou $\beta_2\neq 0$, 
				\begin{equation*}
					\begin{vmatrix}y_n(b) & y_n^\prime(b)\\ y_m(b) & y_m^\prime(b)\end{vmatrix}= 0
				\end{equation*}}
			\item Claramente,
				\begin{equation*}
					p(b)\begin{vmatrix}y_n(b) & y_n^\prime(b)\\ y_m(b) & y_m^\prime(b)\end{vmatrix}-p(a)\begin{vmatrix}y_n(a) & y_n^\prime(a)\\ y_m(a) & y_m^\prime(a)\end{vmatrix}= 0
				\end{equation*}
	      \end{itemize}
      \end{slide}
   
      \begin{slide}[toc=]{Problemas de Sturm-Liouville regulares}
	      \begin{itemize}
		      \item Exemplo: Mostre que para $0\leq x \leq 1$, o problema
			      \begin{align*}
				      y^{\prime\prime}(x) + \lambda y(x) &= 0\\
				      y(0) = 0, y(1) &= 0
			      \end{align*}
			      é um problema de Sturm-Liouville regular.
			      \begin{itemize}
				      \item Primeiro passo: identificar os elementos de equação de Sturm-Liouville:
					      \begin{equation*}
						      \left [ p(x)y^\prime (x)\right ]^\prime + q(x)y(x) + \lambda w(x)y(x) = 0
					      \end{equation*}
			      Nesse caso $p(x) = 1$, $q(x) = 0$, e $w(x) = 1$.\\
		      		      \item Segundo passo: verificar se
					      \begin{itemize}
						      \item $p(x)$, $q(x)$, $w(x)$ são funções reais e contínuas;
						      \item $p(x)$, e $w(x)$ são positivas em um dado intervalo.
					      \end{itemize}
			      \end{itemize}
	      \end{itemize}
      \end{slide}

      \begin{slide}[toc=]{Problemas de Sturm-Liouville regulares}
	      \begin{itemize}
		      \item Exemplo (continuação):
			      \begin{itemize}
				      \item Terceiro passo: verificar de o sistema abaixo possui solução não trivial.
					      \begin{align*}
						      \alpha_1y(a) + \alpha_2 y^\prime (a) &= 0\\
						      \beta_1 y(b) + \beta_2 y^\prime (b) &= 0
					      \end{align*}
					      para $a=0$ e $b= 1$. 
					      Considerando as condições iniciais dadas: $y(0) = 0$, e $ y(1) = 0$, conclui-se que 
					      \begin{align*}
						      \alpha_1 = 1, \quad \alpha_2 = 0\\
						      \beta_1 = 1, \quad \beta_2 = 0
					      \end{align*}
					      ou seja, há solução não trivial para o sistema.
				      \item Conclusão: como todas as condições são satisfeitas, trata-se de um problema de Sturm-Liouville regular.
			      \end{itemize}
	      \end{itemize}
      \end{slide}
   
   \section[ slide = true ]{Problemas de Sturm-Liouville periódicos}
      \begin{slide}[toc=]{Problemas de Sturm-Liouville periódicos}
	      \begin{itemize}
		      \item Seja o intervalo $a\leq x \leq b$, se $p(a) = p(b)$, então as condições de contorno periódicas, 
			      \begin{equation*}
				      y(a) = y(b), \qquad y^\prime (a)= y^\prime (b)
			      \end{equation*}
			      também satisfazem 
				\begin{equation*}
					p(b)\begin{vmatrix}y_n(b) & y_n^\prime(b)\\ y_m(b) & y_m^\prime(b)\end{vmatrix}-p(a)\begin{vmatrix}y_n(a) & y_n^\prime(a)\\ y_m(a) & y_m^\prime(a)\end{vmatrix}= 0
				\end{equation*}
	      \end{itemize}
      \end{slide}
   
   \section[ slide = true ]{Problemas de Sturm-Liouville singulares}
      
\end{document}	
